\documentclass[paper=a4, fontsize=11pt]{scrartcl} % A4 paper and 11pt font size

\usepackage[utf8]{inputenc}
\usepackage[T1]{fontenc} % Use 8-bit encoding that has 256 glyphs
\usepackage{fourier} % Use the Adobe Utopia font for the document - comment this line to return to the LaTeX default
\usepackage{amsmath,amsfonts,amsthm} % Math packages
\usepackage{hyperref}


\usepackage{sectsty} % Allows customizing section commands
\allsectionsfont{\centering \normalfont\scshape} % Make all sections centered, the default font and small caps

\usepackage{fancyhdr} % Custom headers and footers
\pagestyle{fancyplain} % Makes all pages in the document conform to the custom headers and footers
\fancyhead{} % No page header - if you want one, create it in the same way as the footers below
\fancyfoot[L]{} % Empty left footer
\fancyfoot[C]{} % Empty center footer
\fancyfoot[R]{\thepage} % Page numbering for right footer
\renewcommand{\headrulewidth}{0pt} % Remove header underlines
\renewcommand{\footrulewidth}{0pt} % Remove footer underline

\setlength{\headheight}{13.6pt} % Customize the height of the header

\numberwithin{equation}{section} % Number equations within sections (i.e. 1.1, 1.2, 2.1, 2.2 instead of 1, 2, 3, 4)
\numberwithin{figure}{section} % Number figures within sections (i.e. 1.1, 1.2, 2.1, 2.2 instead of 1, 2, 3, 4)
\numberwithin{table}{section} % Number tables within sections (i.e. 1.1, 1.2, 2.1, 2.2 instead of 1, 2, 3, 4)

\setlength\parindent{0pt} % Removes all indentation from paragraphs - comment this line for an assignment with lots of text

\usepackage[utf8]{inputenc}
\usepackage[english]{babel}
\usepackage{amsmath}
\usepackage{mathrsfs}
\usepackage{amssymb}
\usepackage{amsthm}
\usepackage{tikz,tkz-tab}
\usetikzlibrary{datavisualization}
\usetikzlibrary{datavisualization.formats.functions}
\usepackage[a4paper,body=170mm,textheight=250mm]{geometry}
\usepackage{cases}
\newcommand{\ssi}{\Leftrightarrow}
%\newcommand{\N}{\mathbb{N}}
\newcommand{\R}{\mathbb{R}}
\newcommand{\M}{\mathbb{M}}
\newcommand{\K}{\mathbb{K}}
\usepackage{listings}
\newcommand{\Lap}{\mathop{}\!\mathbin\bigtriangleup}
\newcommand{\partielles}[2]{\dfrac{\partial^2 #1}{\partial #2^2}}
\newcommand{\norm}[1]{\left\rVert #1 \right\rVert}
\newcommand{\integrale}[1]{\int_\Omega #1 \: \text{d} \Omega}
\newcommand{\grad}{\overrightarrow{\nabla}}
\newcommand{\arrow}{\overrightarrow}
\newcommand{\ps}[2]{\big\langle #1 , #2 \big \rangle}
\renewcommand{\triangle}{\bigtriangleup}
%\setlength{\parindent}{0pt}
\usepackage{enumitem}
\usepackage{pifont}
\usetikzlibrary{trees}
\newcommand{\EE}[1]{\mathbb{E}\big[ #1 \big]}
\usepackage{bm}

 \usepackage[usenames,dvipsnames]{pstricks}
 \usepackage{epsfig}
 \usepackage{pst-grad} % For gradients
 \usepackage{pst-plot} % For axes
 \usepackage[space]{grffile} % For spaces in paths
 \usepackage{etoolbox} % For spaces in paths
 \makeatletter % For spaces in paths
 \patchcmd\Gread@eps{\@inputcheck#1 }{\@inputcheck"#1"\relax}{}{}

\usepackage{enumitem}
\usepackage{refcount}
\newcounter{count}

\newcommand{\D}{\mathcal{D}}
\newcommand{\N}{\mathcal{N}}
\renewcommand{\sl}{\sigma_\lambda}
\newcommand{\sd}{\sigma_\mathcal{D}}

%----------------------------------------------------------------------------------------
%	TITLE SECTION
%----------------------------------------------------------------------------------------

\newcommand{\horrule}[1]{\rule{\linewidth}{#1}} % Create horizontal rule command with 1 argument of height

\title{	
\normalfont \normalsize 
\textsc{Universite Paris-Saclay - Introduction to Graphical Models} \\ [25pt] % Your university, school and/or department name(s)
\horrule{0.5pt} \\[0.4cm] % Thin top horizontal rule
\huge Homework 1 \\ % The assignment title
\horrule{2pt} \\[0.5cm] % Thick bottom horizontal rule
}

\author{David Obst} % Your name

\date{\normalsize\today} % Today's date or a custom date

\begin{document}

\maketitle

\section{Bayes rule}

Let $\lambda$ and $\D$ two scalar random variables with $p(\lambda) = \N(\lambda,0,\sl^2)$ and $p(\D | \lambda) = \N(\D,\lambda,\sd^2)$ be. The goal is to determine the posterior distribution $p(\lambda | \D)$. \\

Following the Bayes' rule one has:

$$\begin{array}{ccl}

p(\lambda | \D) & = & \dfrac{p(\D | \lambda) p(\lambda)}{p(\D)} \\
				\\
				& \propto & p(\D | \lambda) p(\lambda) \quad \text{with the evidence being only a normalization constant} \\

\end{array}$$

And 

$$\begin{cases}
	p(\lambda) = \dfrac{1}{\sqrt{2\pi \sl^2}} \exp \big\{ - \dfrac{\lambda^2}{2 \sl^2} \big\} \\
	p(\D | \lambda) = \dfrac{1}{\sqrt{2\pi \sd^2}} \exp \big\{ - \dfrac{(\D -\lambda)^2}{2 \sd^2} \big\} 
\end{cases}$$

Therefore by continuing the previous calculation one has:

\begin{equation}
p(\lambda | \D) \propto  \dfrac{1}{\sqrt{(2\pi)^2\sd^2 \sl^2}} \exp \Big( \dfrac{-1}{2} \big( \dfrac{(\D - \lambda)^2}{\sd^2} + \dfrac{\lambda^2}{\sl^2} \big) \Big)
\label{eq1}
\end{equation}


Let $T(\D,\lambda) = \dfrac{(\D - \lambda)^2}{\sd^2} + \dfrac{\lambda^2}{\sl^2} $ be the inner term of the exponential (without the factor $-1/2$). Then by developping it is rewritten: \\

$
\begin{array}{ccl}

T(\D,\lambda) & = & \dfrac{\D^2 - 2\lambda \D + \lambda^2}{\sd^2} + \dfrac{\lambda^2}{\sl^2} \\
\\
			  & = & \Big( \dfrac{1}{\sd^2} + \dfrac{1}{\sl^2} \Big) \lambda^2 - \dfrac{2 \D}{\sd^2} \lambda + \dfrac{1}{\sd^2} \D

\end{array}
$

Let $\sigma$ defined by $\dfrac{1}{\sigma^2} = \dfrac{1}{\sd^2} + \dfrac{1}{\sl^2} $ be. Continuining the calculations yields:

$
\begin{array}{ccl}

T(\D,\lambda) & = & \dfrac{1}{\sigma^2} \lambda - 2 \dfrac{\D}{\sd^2} \lambda + \dfrac{\sigma^2}{\sd^2} d^2 \\
\\
  & = & \dfrac{1}{\sigma^2} \Big( \lambda - \dfrac{\sigma}{\sd} \D \Big)^2 - \dfrac{\D}{\sd^2}

\end{array}
$

Consequently, replacing this form in \ref{eq1} yields:

$
\begin{array}{ccl}

p(\lambda | \D) & \propto & \dfrac{1}{\sqrt{(2\pi)^2 \sd^2 \sl^2}} \exp \Big( -\dfrac{1}{2\sigma^2} \big( \lambda - \dfrac{\sigma}{\sd} \D \big)^2 \Big) \exp \big( \dfrac{\D}{2 \sd^2} \big)

\end{array}
$

Therefore we have 

\begin{equation}
	p(\lambda | \D) \propto \exp \Big( - \dfrac{1}{2 \sigma^2} \big( \lambda - \dfrac{\sigma}{\sd} \D \big)^2 \Big)
\end{equation}

Therefore the posterior $\lambda | \D$ follows a normal distribution:

\begin{equation}
\boxed{p(\lambda | \D) = \N(\mu',\sigma'^2) \quad \text{with} \quad \begin{cases}
	\mu' = \dfrac{\sl}{\sqrt{\sd^2 + \sl^2}} \D\\
	\\
	\sigma'^2 = \dfrac{\sd^2 \sl^2}{\sd^2 + \sl^2}
\end{cases} } 
\end{equation}

\bigskip

\section{Graphical Models}

\begin{list}{\textbf{Question \arabic{count} -}}{\usecounter{count}}
	\item The Directed Acyclic Graph (DAG) is obtained by reading the factorization of the joint distribution.
\end{list}

\begin{figure}[h]
\centering
\psscalebox{1.0 1.0} % Change this value to rescale the drawing.
{
\begin{pspicture}(0,-5.8)(11.2,5.8)
\pscircle[linecolor=black, linewidth=0.04, dimen=outer](0.8,5.0){0.8}
\pscircle[linecolor=black, linewidth=0.04, dimen=outer](10.4,5.0){0.8}
\pscircle[linecolor=black, linewidth=0.04, dimen=outer](0.8,1.8){0.8}
\pscircle[linecolor=black, linewidth=0.04, dimen=outer](5.6,1.8){0.8}
\pscircle[linecolor=black, linewidth=0.04, dimen=outer](10.4,1.8){0.8}
\psline[linecolor=black, linewidth=0.04, arrowsize=0.05291667cm 2.0,arrowlength=1.4,arrowinset=0.0]{->}(0.8,4.2)(0.8,2.6)
\psline[linecolor=black, linewidth=0.04, arrowsize=0.05291667cm 2.0,arrowlength=1.4,arrowinset=0.0]{->}(10.4,4.2)(10.4,2.6)
\pscircle[linecolor=black, linewidth=0.04, dimen=outer](3.2,-1.8){0.8}
\pscircle[linecolor=black, linewidth=0.04, dimen=outer](10.4,-1.8){0.8}
\psline[linecolor=black, linewidth=0.04, arrowsize=0.05291667cm 2.0,arrowlength=1.4,arrowinset=0.0]{->}(4.0,-1.8)(9.6,-1.8)
\psline[linecolor=black, linewidth=0.04, arrowsize=0.05291667cm 2.0,arrowlength=1.4,arrowinset=0.0]{->}(3.2,-2.6)(3.2,-4.2)
\pscircle[linecolor=black, linewidth=0.04, dimen=outer](3.2,-5.0){0.8}
\psline[linecolor=black, linewidth=0.04, arrowsize=0.05291667cm 2.0,arrowlength=1.4,arrowinset=0.0]{->}(10.4,1.0)(10.4,-1.0)
\rput[bl](0.56,4.78){\huge A}
\rput[bl](0.58,1.6){\huge T}
\rput[bl](10.08,4.74){\huge M}
\rput[bl](5.44,1.58){\huge L}
\rput[bl](10.18,1.56){\huge B}
\rput[bl](10.2,-2.04){\huge D}
\rput[bl](3.04,-2.0){\huge F}
\rput[bl](2.96,-5.22){\huge X}
\psline[linecolor=black, linewidth=0.04, arrowsize=0.05291667cm 2.0,arrowlength=1.4,arrowinset=0.0]{->}(5.6,1.0)(3.6,-1.0)
\psline[linecolor=black, linewidth=0.04, arrowsize=0.05291667cm 2.0,arrowlength=1.4,arrowinset=0.0]{->}(0.8,1.0)(2.8,-1.0)
\psline[linecolor=black, linewidth=0.04, arrowsize=0.05291667cm 2.0,arrowlength=1.4,arrowinset=0.0]{->}(9.76,4.52)(6.3,2.2)
\end{pspicture}
}
\caption{Question 1: DAG of the distribution}
\end{figure}


\end{document}
